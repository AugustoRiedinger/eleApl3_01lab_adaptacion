% ****************************************************************************************************
% classicthesis-config-LyX.tex
% Use it in the preamble of your ClassicThesis.lyx
% ****************************************************************************************************
% If you like the classicthesis, we would appreciate a postcard.
% André's address can be found in the file ClassicThesis.pdf. A collection
% of the postcards received so far is available online at
% http://postcards.miede.de
% ****************************************************************************************************

% ****************************************************************************************************
%                         		C O L O R S 
%****************************************************************************************************



% ****************************************************************************************************
% 1. Configure classicthesis for your needs here. At some point you will probably want to set 
% "drafting" to false in order to deactivate the time-stamp on the bottom of the pages
% See ClassicThesis.pdf for more information.
% ****************************************************************************************************
\PassOptionsToPackage{
  drafting=false,      % print version information on the bottom of the pages
  tocaligned=false,   % the left column of the toc will be aligned (no indentation)
  dottedtoc=false,    % page numbers in ToC flushed right
  parts=true,         % use part division
  eulerchapternumbers=false, % use AMS Euler for chapter Number font (otherwise Palatino)
  linedheaders=true,       % chaper headers will have line above and beneath
  floatperchapter=true,     % numbering per chapter for all floats (i.e., Figure 1.1)
  eulermath=false,    % use awesome Euler fonts for mathematical formulae (only with pdfLaTeX)
  beramono=false,      % toggle a nice monospaced font (w/ bold)
  palatino=false,      % toggle the standard roman font (the font of the paragraphs), see end of this file for more suggestions
  style=arsclassica % classicthesis, arsclassica
}{classicthesis}


% Martin: Resize gather equations
\usepackage{resizegather}
% ****************************************************************************************************
% 2. Personal data and user ad-hoc commands
% ****************************************************************************************************
\newcommand{\myTitle}{NOTES ON FLUID STRUCTURE INTERACTION\xspace}
\newcommand{\mySubtitle}{Grupo de Investigación en Multifísica Aplicada\xspace}
\newcommand{\myDegree}{Dr.\xspace}
\newcommand{\myName}{Martín Saravia\xspace}
\newcommand{\myProf}{Put name here\xspace}
\newcommand{\myOtherProf}{Put name here\xspace}
\newcommand{\mySupervisor}{Put name here\xspace}
\newcommand{\myFaculty}{Put data here\xspace}
\newcommand{\myDepartment}{Put data here\xspace}
\newcommand{\myUni}{Put data here\xspace}
\newcommand{\myLocation}{Saarbrücken\xspace}
\newcommand{\myTime}{June 2019\xspace}
\newcommand{\myVersion}{Version 0.0}

% ****************************************************************************************************
% Martin: Load some packages
\usepackage{minted} % should be loaded before scrhack otherwise it does not work. 
\setminted[python]{style = fruity, frame=lines,  bgcolor=MaterialGrey800, fontsize=\footnotesize, linenos=true, tabsize=4 , beameroverlays=true}


% ****************************************************************************************************
% 3. Loading some handy packages
% ****************************************************************************************************
\usepackage{csquotes} % biblatex depends on these
\usepackage{scrhack} % fix warnings when using KOMA with listings package
\usepackage{xspace} % to get the spacing after macros right
\usepackage{mparhack} % get marginpar right

% Martin: Epigraph configuration
\usepackage{epigraph} 
\setlength\epigraphwidth{.95\textwidth} % change the width
\renewcommand{\textflush}{flushepinormal} %justify

\PassOptionsToPackage{printonlyused,smaller}{acronym}
  \usepackage{acronym} % nice macros for handling all acronyms in the thesis
  %\renewcommand{\bflabel}[1]{{#1}\hfill} % fix the list of acronyms --> no longer working
  %\renewcommand*{\acsfont}[1]{\textsc{#1}} 
  %\renewcommand*{\aclabelfont}[1]{\acsfont{#1}}
  \def\bflabel#1{{\acsfont{#1}\hfill}}
  \def\aclabelfont#1{\acsfont{#1}}
% ****************************************************************************************************


% ****************************************************************************************************
% 4. Setup floats: tables, (sub)figures, and captions
% ****************************************************************************************************
\usepackage{tabularx} % better tables
  \setlength{\extrarowheight}{3pt} % increase table row height
\newcommand{\tableheadline}[1]{\multicolumn{1}{l}{\spacedlowsmallcaps{#1}}}
\usepackage{subfig}
% ****************************************************************************************************


% ****************************************************************************************************
% 5. Setup code listings
% ****************************************************************************************************
\usepackage{listings}
%\lstset{emph={trueIndex,root},emphstyle=\color{BlueViolet}}%\underbar} % for special keywords
\lstset{language=[LaTeX]Tex,%C++,
  keywordstyle=\color{RoyalBlue},%\bfseries,
  basicstyle=\small\ttfamily,
  %identifierstyle=\color{NavyBlue},
  commentstyle=\color{Green}\ttfamily,
  stringstyle=\rmfamily,
  numbers=none,%left,%
  numberstyle=\scriptsize,%\tiny
  stepnumber=5,
  numbersep=8pt,
  showstringspaces=false,
  breaklines=true,
  frameround=ftff,
  frame=single,
  belowcaptionskip=.75\baselineskip
  %frame=L
}

% Martin: Delete the blank page after parts (does not work)
\makeatletter
\def\@endpart{\vfil\newpage}
\makeatother

% ****************************************************************************************************


% ****************************************************************************************************
% 6. Last calls before the bar closes
% ****************************************************************************************************
% ********************************************************************
% Her Majesty herself
% ********************************************************************
\usepackage{classicthesis}


% ********************************************************************
% Fine-tune hyperreferences (hyperref should be called last)
% ********************************************************************
\hypersetup{%
  %draft, % hyperref's draft mode, for printing see below
  colorlinks=true, linktocpage=true, pdfstartpage=3, pdfstartview=FitV,%
  % uncomment the following line if you want to have black links (e.g., for printing)
  %colorlinks=false, linktocpage=false, pdfstartpage=3, pdfstartview=FitV, pdfborder={0 0 0},%
  breaklinks=true, pdfpagemode=UseNone, pageanchor=true, pdfpagemode=UseNone,%
  plainpages=false, bookmarksnumbered, bookmarksopen=true, bookmarksopenlevel=1,%
  hypertexnames=true, pdfhighlight=/O,%nesting=true,%frenchlinks,%
  urlcolor=CTurl, linkcolor=CTlink, citecolor=CTcitation,%
  %urlcolor=Black, linkcolor=Black, citecolor=Black,%
  pdftitle={\myTitle},%
  pdfauthor={\textcopyright\ \myName, \myUni, \myFaculty},%
  pdfsubject={},%
  pdfkeywords={},%
  pdfcreator={},%
  pdfproducer={LaTeX with classicthesis style}%
}



% ********************************************************************
% Setup autoreferences (hyperref and babel)
% ********************************************************************
% There are some issues regarding autorefnames
% http://www.ureader.de/msg/136221647.aspx
% http://www.tex.ac.uk/cgi-bin/texfaq2html?label=latexwords
% you have to redefine the macros for the
% language you use, e.g., american, ngerman
% (as chosen when loading babel/AtBeginDocument)
% ********************************************************************
\makeatletter
\@ifpackageloaded{babel}%
  {%
    \addto\extrasamerican{%
      \renewcommand*{\figureautorefname}{Figure}%
      \renewcommand*{\tableautorefname}{Table}%
      \renewcommand*{\partautorefname}{Part}%
      \renewcommand*{\chapterautorefname}{Chapter}%
      \renewcommand*{\sectionautorefname}{Section}%
      \renewcommand*{\subsectionautorefname}{Section}%
      \renewcommand*{\subsubsectionautorefname}{Section}%
    }%
    \addto\extrasngerman{%
      \renewcommand*{\paragraphautorefname}{Absatz}%
      \renewcommand*{\subparagraphautorefname}{Unterabsatz}%
      \renewcommand*{\footnoteautorefname}{Fu\"snote}%
      \renewcommand*{\FancyVerbLineautorefname}{Zeile}%
      \renewcommand*{\theoremautorefname}{Theorem}%
      \renewcommand*{\appendixautorefname}{Anhang}%
      \renewcommand*{\equationautorefname}{Gleichung}%
      \renewcommand*{\itemautorefname}{Punkt}%
    }%
      % Fix to getting autorefs for subfigures right (thanks to Belinda Vogt for changing the definition)
      \providecommand{\subfigureautorefname}{\figureautorefname}%
    }{\relax}
\makeatother

% ****************************************************************************************************
% 7. Further adjustments (experimental)
% ****************************************************************************************************
% ********************************************************************
% Changing the text area
% ********************************************************************
\areaset[current]{400pt}{761pt} % 686 (factor 2.2) + 33 head + 42 head \the\footskip
\setlength{\marginparwidth}{7em}%
\setlength{\marginparsep}{2em}%

% ********************************************************************
% Using different fonts
% This is for pdflatex; xelatex and lualatex have their own way
% This settings overwrite some of the options in the begginning, i.e. beramono=True
% ********************************************************************
% Martin font setting
\usepackage[osf]{libertine}  % Set the general font
%\usepackage[scaled=0.82]{beramono}  % Beramono for typewriter font (mono)
\renewcommand{\ttdefault}{cmtt}      % Computer Modern Typewriter for typewriter font(mono)


%\usepackage[oldstylenums]{kpfonts} % oldstyle notextcomp
%\usepackage[light,condensed,math]{iwona}
%\renewcommand{\sfdefault}{iwona}
%\usepackage{lmodern} % <-- no osf support :-(
%\usepackage{cfr-lm} %
%\usepackage[urw-garamond]{mathdesign} %<-- no osf support :-(
%\usepackage[default,osfigures]{opensans} % scale=0.95
%\usepackage[sfdefault]{FiraSans}
%\usepackage[opticals,mathlf]{MinionPro} % onlytext
%\renewcommand{\rmdefault}{artemisia}
% ********************************************************************
% \usepackage[largesc,osf]{newpxtext}
%\linespread{1.05} % a bit more for Palatino
% Used to fix these:
% https://bitbucket.org/amiede/classicthesis/issues/139/italics-in-pallatino-capitals-chapter
% https://bitbucket.org/amiede/classicthesis/issues/45/problema-testatine-su-classicthesis-style
% ********************************************************************
\linespread{1} % Martin: Distance between lines (for palatino use 1.05, the default)

% Martin; Change the space between items of itemize
\usepackage{xpatch}
\xpatchcmd{\itemize}
  {\def\makelabel}
  {\setlength{\itemsep}{0pt}\def\makelabel}
  {}
  {}
% ****************************************************************************************************
%     SET lmodern style only for equations, comment  \usepackage{lmodern} to maintain Palatino for the text
% ****************************************************************************************************
\usepackage{amsmath,amsfonts,amsthm,bm}
\SetSymbolFont{operators}   {normal}{OT1}{cmr} {m}{n}
\SetSymbolFont{letters}     {normal}{OML}{cmm} {m}{it}
\SetSymbolFont{symbols}     {normal}{OMS}{cmsy}{m}{n}
\SetSymbolFont{largesymbols}{normal}{OMX}{cmex}{m}{n}
\SetSymbolFont{operators}   {bold}  {OT1}{cmr} {bx}{n}
\SetSymbolFont{letters}     {bold}  {OML}{cmm} {b}{it}
\SetSymbolFont{symbols}     {bold}  {OMS}{cmsy}{b}{n}
\SetSymbolFont{largesymbols}{bold}  {OMX}{cmex}{m}{n}

\SetMathAlphabet{\mathbf}{normal}{OT1}{cmr}{bx}{n}
\SetMathAlphabet{\mathsf}{normal}{OT1}{cmss}{m}{n}
\SetMathAlphabet{\mathit}{normal}{OT1}{cmr}{m}{it}
\SetMathAlphabet{\mathtt}{normal}{OT1}{cmtt}{m}{n}
\SetMathAlphabet{\mathbf}{bold}  {OT1}{cmr}{bx}{n}
\SetMathAlphabet{\mathsf}{bold}  {OT1}{cmss}{bx}{n}
\SetMathAlphabet{\mathit}{bold}  {OT1}{cmr}{bx}{it}
\SetMathAlphabet{\mathtt}{bold}  {OT1}{cmtt}{m}{n}

% ****************************************************************************************************


\usepackage{xcolor-material}
\colorlet{MaterialTitle}{MaterialDeepOrangeA700}
\colorlet{MaterialNumber}{MaterialDeepOrangeA700}
\colorlet{MaterialExample}{MaterialDeepOrangeA700}

\colorlet{CTtitle}{MaterialTitle}
\colorlet{CTsemi}{MaterialTitle}


%\newcounter{example}

%\def\exampletext{Example} % If English

%\NewDocumentEnvironment{example}{ O{} }
%{
%\newtcolorbox[use counter=example]{examplebox}{%
%    % Example Frame Start
%    empty,% Empty previously set parameters
%    title={\exampletext: #1},% use \thetcbcounter to access the example counter text
%    % Attaching a box requires an overlay
%    attach boxed title to top left,
%       % Ensures proper line breaking in longer titles
%    minipage boxed title,
%    % (boxed title style requires an overlay)
%    boxed title style={empty,size=minimal,toprule=0pt,top=4pt,left=3mm,overlay={}},
%    coltitle=MaterialExample,
%    fonttitle=\bfseries,
%    before=\par\medskip\noindent,parbox=false,boxsep=0pt,left=3mm,right=0mm,top=2pt,breakable,pad at break=0mm,
%       before upper=\csname @totalleftmargin\endcsname0pt, % Use instead of parbox=true. This ensures parskip is inherited by box.
%    % Handles box when it exists on one page only
%    overlay unbroken={\draw[MaterialExample,line width=.5pt] ([xshift=-0pt]title.north west) -- ([xshift=-0pt]frame.south west); },
%    % Handles multipage box: first page
%    overlay first={\draw[MaterialExample,line width=.5pt] ([xshift=-0pt]title.north west) -- ([xshift=-0pt]frame.south west); },
%    % Handles multipage box: middle page
%    overlay middle={\draw[MaterialExample,line width=.5pt] ([xshift=-0pt]frame.north west) -- ([xshift=-0pt]frame.south west); },
%    % Handles multipage box: last page
%    overlay last={\draw[MaterialExample,line width=.5pt] ([xshift=-0pt]frame.north west) -- ([xshift=-0pt]frame.south west); },%
%    }
%\begin{examplebox}}
%{\end{examplebox}\endlist}


\usepackage[most]{tcolorbox}

\newtcolorbox{exa}[1]{%
	boxrule=0pt,
	fontupper=\itshape,
	frame hidden,
	sharp corners,
	enhanced,
	borderline west={1pt}{0pt}{MaterialExample}
}



